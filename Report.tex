%%%%%%%%%%%%%%%%%%%%%%%%%%%%%%%%%%%%%%%%%
% Large Colored Title Article
% LaTeX Template
% Version 1.0 (15/8/12)
%
% This template has been downloaded from:
% http://www.LaTeXTemplates.com
%
% Original author:
% Frits Wenneker (http://www.howtotex.com)
%
% License:
% CC BY-NC-SA 3.0 (http://creativecommons.org/licenses/by-nc-sa/3.0/)
%
%%%%%%%%%%%%%%%%%%%%%%%%%%%%%%%%%%%%%%%%%

%----------------------------------------------------------------------------------------
%	PACKAGES AND OTHER DOCUMENT CONFIGURATIONS
%----------------------------------------------------------------------------------------

\documentclass[DIV=calc, paper=a4, fontsize=10pt, twocolumn]{scrartcl}	 % A4 paper and 11pt font size

\usepackage{lipsum} % Used for inserting dummy 'Lorem ipsum' text into the template
%\usepackage[english]{babel} % English language/hyphenation
\usepackage[protrusion=true,expansion=true]{microtype} % Better typography
\usepackage{amsmath,amsfonts,amsthm} % Math packages
\usepackage[svgnames]{xcolor} % Enabling colors by their 'svgnames'
\usepackage[hang, small,labelfont=bf,up,textfont=it,up]{caption} % Custom captions under/above floats in tables or figures
\usepackage{booktabs} % Horizontal rules in tables
\usepackage{fix-cm}	 % Custom font sizes - used for the initial letter in the document

\usepackage{sectsty} % Enables custom section titles
%\allsectionsfont{\usefont{OT1}{phv}{b}{n}} % Change the font of all section commands

\usepackage{fancyhdr} % Needed to define custom headers/footers
\pagestyle{fancy} % Enables the custom headers/footers
\usepackage{lastpage} % Used to determine the number of pages in the document (for "Page X of Total")

% Headers - all currently empty
\lhead{}
\chead{}
\rhead{}

% Footers
\lfoot{}
\cfoot{}
\rfoot{\footnotesize Page \thepage\ of \pageref{LastPage}} % "Page 1 of 2"

\renewcommand{\headrulewidth}{0.0pt} % No header rule
\renewcommand{\footrulewidth}{0.4pt} % Thin footer rule

\usepackage{lettrine} % Package to accentuate the first letter of the text
\newcommand{\initial}[1]{ % Defines the command and style for the first letter
\lettrine[lines=3,lhang=0.3,nindent=0em]{
\color{DarkGoldenrod}
{\textsf{#1}}}{}}


\usepackage{enumerate}
%\usepackage[dvipdfm]{graphicx}
\usepackage{fontspec}
\usepackage{xunicode}
\usepackage{xltxtra}
\usepackage{xgreek}

%SET THE FONTS HERE
\setmainfont[Mapping=tex-text]{Open Sans}
\setsansfont[Mapping=tex-text]{Open Sans}
\setmonofont[Mapping=tex-text]{Open Sans}

%CREATE A N W ENVIRONMENT FOR ITEMIZE & ENUMERATION TO SAVE SPACE
\newenvironment{packed_item}{
\begin{itemize}
  \setlength{\itemsep}{0pt}
  \setlength{\parskip}{0pt}
  \setlength{\parsep}{0pt}
}{\end{itemize}}

\newenvironment{packed_enum}{
\begin{enumerate}
 \setlength{\itemsep}{0pt}
 \setlength{\parsep}{0pt}
 \setlength{\topsep}{0pt}
 \setlength{\partopsep}{0pt}
 \setlength{\leftmargin}{0.5em}
 \setlength{\labelwidth}{0.5em}
 \setlength{\labelsep}{0.5em} 
}{\end{enumerate}}


\usepackage{fancyhdr}
\pagestyle{fancy}

\lhead{\footnotesize {Τίτλος} }
\lfoot{\footnotesize {\copyright  Όνομα Επώνυμο}}
\cfoot{}
\rhead{\footnotesize {\textbf{Υπότιτλος} :  \today}}
%\cfoot{\footnotesize \today}
\rfoot{\footnotesize Σελίδα \thepage\ από \pageref{LastPage}}
\renewcommand{\headheight}{24pt}
\renewcommand{\footrulewidth}{0.4pt}

%\renewcommand\headrule{\vspace{-8pt}}
%\setlanguage{monogreek}

%\usepackage[utf8x]{inputenc}
%\usepackage[greek]{babel}
\usepackage{listings}
%\lstloadlanguages{C}
\lstdefinelanguage{rock}
{morekeywords={if, elif, done, else, in, fi, case, esac, then, for , do, while, until, break, select, continue},
sensitive=false,
morecomment=[l]{//},
morecomment=[s]{/*}{*/},
morestring=[b]",
}
\lstset{language=rock,commentstyle=\scriptsize,frameround=fttt,basicstyle=\ttfamily,}
%rg
\usepackage{color}
\usepackage{textcomp}
\definecolor{listinggray}{gray}{0.9}
\definecolor{lbcolor}{rgb}{0.9,0.9,0.9}
\lstset{
	backgroundcolor=,
	tabsize=4,
	rulecolor=,
	language=rock,
        upquote=true,
        aboveskip={1.0\baselineskip},
        columns=fixed,
        showstringspaces=false,
        extendedchars=true,
        breaklines=false,
        prebreak = \raisebox{0ex}[0ex][0ex]{\ensuremath{\hookleftarrow}},
        frame=lines,
        showtabs=false,
        showspaces=false,
        showstringspaces=false,
        identifierstyle=\ttfamily,
	basicstyle=\scriptsize,
        keywordstyle=\color[rgb]{0,0,1},
        commentstyle=\color[rgb]{0.133,0.545,0.133},
        stringstyle=\color[rgb]{0.627,0.126,0.941},
}

\usepackage{lastpage}
\usepackage{epigraph}

\usepackage{hyperref}
\usepackage[compact]{titlesec}
\titlespacing*{\section}{10pt}{10pt}{5pt}
\usepackage{verbatim} 

\usepackage{tabularx}

\usepackage{multirow}
\usepackage{colortbl}
%----------------------------------------------------------------------------------------
%	TITLE SECTION
%----------------------------------------------------------------------------------------

\usepackage{titling} % Allows custom title configuration

\newcommand{\HorRule}{\color{DarkGoldenrod} \rule{\linewidth}{1pt}} % Defines the gold horizontal rule around the title

\pretitle{\vspace{-30pt} \begin{flushleft} \HorRule \fontsize{30}{30}  \color{DarkRed} \selectfont} % Horizontal rule before the title

\title{Τίτλος Αναφοράς} % Your article title

\posttitle{\par\end{flushleft}} % Whitespace under the title

\preauthor{\begin{flushleft}\large  \color{DarkRed}} % Author font configuration

\author{Όνομα Επώνυμο } % Your name

\postauthor{\footnotesize \color{Black} % Configuration for the institution name
\href{http://www.dit.hua.gr}{Τμήμα Πληροφορικής \& Τηλεματικής} % Your institution

\par\end{flushleft}\HorRule} % Horizontal rule after the title

\date{} % Add a date here if you would like one to appear underneath the title block

%----------------------------------------------------------------------------------------

\begin{document}

\maketitle % Print the title

\thispagestyle{fancy} % Enabling the custom headers/footers for the first page 

%----------------------------------------------------------------------------------------
%	ABSTRACT
%----------------------------------------------------------------------------------------

% The first character should be within \initial{}
\initial{Ε}\textbf{δώ πηγαίνει η περίληψη. Αυτη η παράγραφος έχει την περίληψη του κειμένου που ακολουθεί. Θα παρατηρήσετε ότι έχει το πρώτο γράμμα κεφαλαίο. }

%----------------------------------------------------------------------------------------
%	ARTICLE CONTENTS
%----------------------------------------------------------------------------------------

%EPIGRAPH
\epigraphhead[10]{\epigraph{"UNIX is basically a simple operating system, but you have to be a genius to understand the
simplicity."}{\textit{\href{http://en.wikipedia.org/wiki/Dennis_Ritchie}{Dennis Ritchie}}}}

\section{Ενότητα 1}

Ας δούμε λίγο κώδικα C.

\begin{lstlisting}
#include <stdio.h>
#define READ 0 /* Read end of pipe */
#define WRITE 1 /* Write end of pipe */
char *phrase="This is a test phrase";

main() {
int pid, fd[2], bytes;
char message[100];
if (pipe(fd)==-1) {   /*create a pipe*/
	perror("pipe");
	exit(1); }
if ((pid=fork())==-1) {  /*fork a child*/
	perror("fork");
	exit(1); }
if (pid==0) {	/*child writer */
	close(fd[READ]);	/*close unused end */
	write(fd[WRITE], phrase, strlen(phrase)+1);
	close(fd[WRITE]);  }/*close used end */
else {		/* parent reader */
	close(fd[WRITE]);	/*close unused end */
	bytes=read(fd[READ], message, sizeof(message));
	printf("Read %d bytes : %s\n", bytes, message);
	close(fd[READ]); }	 /*close used end */
}
\end{lstlisting}


\section{Ενότητα 2}

Οι υποδοχές (sockets) παρέχουν point-to-point two way communication μεταξύ δυο διεργασιών. Αποτελούν έναν από τους βασικούς τρόπους
επικοινωνίας μεταξύ διεργασιών, οι οποίες μπορεί να βρίσκονται στο ίδιο μηχάνημα ή ακόμη και σε διαφορετικά. Μια υποδοχή αποτελεί το άκρο
της επικοινωνίας και το οποίο μπορεί να έχει όνομα. Υπάρχουν διαφορετικοί τύποι υποδοχών.

Οι υποδοχές υπάρχουν μέσα σε «πεδία επικοινωνίας» τα λεγόμενα \emph{communication domains}. Ένα socket domain καθορίζει το περιβάλλον της
επικοινωνίας, το οποίο παρέχει μια δομή διευθυνσιοδότησης (addressing structure) και ένα σύνολο από πρωτόκολλα.

Τα πιο συνηθισμένα communication domains είναι το \emph{UNIX domain} και το \emph{Internet domain} \footnote{Τα πιο γνωστά communication
domains είναι: AF\_INET (IPv4 Internet domain) , AF\_INET6 (IPv6 Internet domain), AF\_UNIX (UNIX domain), AF\_UNSPEC (unspecified)}.

\begin{itemize}
 \item \emph{UNIX domain}: παρέχει διευθυνσιοδότηση υποδοχέων μέσα σε ένα σύστημα. Τα unix domain sockets ονομάζονται με Unix paths. 
 \item \emph{internet domain}: καθορίζεται από τη IP διεύθυνση του μηχανήματος που το φιλοξενεί και τον αριθμό πόρτας (συνήθως ένας αριθμός
πάνω από 2000 μέχρι $2^{16}$) 
\end{itemize}




 
Ας δούμε μια εικόνα 
\begin{figure}[ht] 
\centering
\scalebox{0.3}{\includegraphics{latex.png}}
\caption{Το \LaTeX\ έχει πλάκα!}
\label{endianess}
\end{figure}


Ας δούμε έναν πίνακα

\begin{center}
\begin{table}[h]
\small
\begin{tabularx}{\textwidth}{l|l|l}
\rowcolor[gray]{0.9}
  \textbf{Σήμα} & \textbf{Κωδικός} &  \textbf{Περιγραφή} \\
SIGHUP  &1	&Hang up\\
SIGINT	&2	& (Ctrl + C)\\
SIGQUIT	&3	& (Ctrl + \textbackslash)\\
SIGKILL	&9	&quit immediately\\
SIGALRM	&14	&Alarm Clock signal (used for timers)\\
SIGTERM	&15	&Software termination signal \\
SIGSTOP	&17	&Stop signal\\
SIGTSTP	&18	& (Ctrl + Z)\\
SIGCONT	&19	&A stopped process is being continued\\
\end{tabularx}  
\caption{τα πιο συνηθισμένα σήματα στο unix} 
\label{tab:signals}          
\end{table}
\end{center}

\end{document}
